\documentclass[../midgard.tex]{subfiles}
\graphicspath{{\subfix{../images/}}}
\begin{document}

\section{Data availability layer}
\label{h:data-availability-layer}

The data availability layer is critical to Midgard's security because every committed block needs to be publicly available throughout the maturity period so that watchers can detect and prove fraud before invalid blocks are merged.

We are considering three solutions for the data availability layer (in decreasing preference):
\begin{enumerate}
  \item Cardano Leios blobs (the ideal solution)
  \item Multi-signature committees
\end{enumerate}

\subsection{Data availability via Leios blobs}
\label{h:data-availability-leios}

The ideal data availability solution for Midgard is based on Cardano Leios blobs, which are a proposed feature in Cardano that will support large-scale transient data storage secured via L1 consensus.

The contents of these blobs do not have to be verified by Cardano's Ouroboros consensus protocol, and they only have to be stored for up to 30 days.
This means that a large amount of data can be stored in these blobs sustainably for a low cost, which we expect to be several orders of magnitude lower than Cardano's cost for transaction metadata (which is stored permanently).

Leios blobs are a natural intermediate point on Leios' multi-year roadmap toward full input-endorser capabilities.
We believe that they are achievable before Midgard's planned deployment on mainnet, and we will support the Leios team to help bring them to Cardano sooner.

In the Leios-based data availability solution for Midgard, operators will pay to store their full non-Merkelized blocks inside Leios blobs for the full maturity period.
Leios itself will provide timestamps and (non-Merkle) hashes for the blobs and ensure that the blob contents are accessible.
Midgard's L1 smart contracts will be able to access the timestamps and non-Merkle hashes of blobs directly.

The block data inside each Leios blob will be sufficient to be converted offchain into the Merkelized representation of Midgard blocks that is necessary to construct fraud proofs.
The correspondence between the non-Merkelized block data stored in the Leios blob and the Merkle root hash declared in the block header will be verifiable via a special fraud proof verification procedure.
This procedure will calculate the Merkle root hash in a streaming fashion over the block data and compare it to the declared Merkle root hash in the block header.

\notebox{We hope to further streamline this Merkle root hash verification procedure by collaborating with the Cardano Leios and Plutus teams to make it a Plutus builtin operation.
  This builtin can be much more expensive than typical Plutus built-ins, and Midgard fraud provers will be happy to pay the cost because the stakes of DA fraud and the reward for proving it are much higher.
  Nonetheless, this will not impose an ongoing cost burden on Midgard because this verification procedure will not need to be invoked during normal operation with honest blocks, and invalid blocks will be rare.
}

Operators' costs for storing blocks in Leios will be offset by the revenue they collect from Midgard transaction, deposit, and withdrawal fees.
Furthermore, the Leios blob storage fees will become an additional source of revenue for Cardano L1 block producing nodes, further boosting the economic security of Cardano L1 on which Midgard depends.

\subsection{Data availability via Mithril}
\label{h:data-availability-multisig}

If Leios blobs are not available on Cardano mainnet in time for Midgard's deployment, a viable alternative is using stake-weighted Mithril certificates to ensure data availability.

Mithril was specifically designed to address the data-availability problem, as stated in the Mithril whitepaper:
\begin{quote}
Mithril provides an immediate solution to the data-availability problem: if the underlying consensus protocol is run on references for which a Mithril signature exists, data availability is (cryptographically) guaranteed
\end{quote}

In this approach, a state commitment is only considered valid if it is accompanied by a Mithril certificate signed by a stake-weighted quorum of Mithril participants. This certificate serves as a cryptographic guarantee that the necessary block data is publicly available.

The process follows these steps:
\begin{enumerate}
  \item \textbf{Publishing Data:} Operators must publish their full block data at a publicly accessible location. This data must be sufficient to reconstruct the Merkle roots referenced by the state commitment.
  \item \textbf{Verification by Mithril Participants:} Each Mithril participant independently retrieves the published data and verifies that:
  \begin{itemize}
    \item The data is available.
    \item It corresponds exactly to the Merkle roots in the state commitment.
  \end{itemize}
  \item \textbf{Signing the Certificate:} If the data meets these conditions, the Mithril participant signs the Mithril certificate.
  \item \textbf{Onchain Verification:} The state commitment cannot be appended to the Midgard state queue until:
  \begin{itemize}
    \item An associated Mithril certificate has been produced.
    \item The stake-weight attributed to the certificate exceeds Midgard's \code{da\_multisig\_threshold} parameter.
    \item The message attested to by the certificate uniquely identifies the state commitment.
  \end{itemize}
  \item \textbf{Appending to the State Queue:} Only once all the above conditions are met will the operator be able to append the state commitment to the Midgard state queue.
\end{enumerate}

By leveraging Mithril, Midgard ensures that state commitments are only accepted when their corresponding block data is provably available. This mechanism safeguards Midgard against data-availability fraud, where a malicious operator attempts to submit a state commitment without disclosing the underlying data. Without access to this data, fraud provers would be unable to construct and submit fraud proofs, allowing the fraudulent state commitment to become canonical once the fraud detection window closes. By requiring a Mithril certificate before a state commitment is appended, Midgard guarantees that the data remains accessible, preserving the integrity of the system.

A key difference between the Leios blob and Mithril-based data availability (DA) solutions lies in the security guarantees provided by each approach. In the Leios blob solution, data availability is assured with the full security of Cardano’s consensus, meaning that all active stake on the network contributes to securing DA, just as it does for transaction finality and block production. This ensures that any attempt to suppress or manipulate stored data would require an adversary to control a majority of Cardano’s total stake, making attacks highly impractical. In contrast, the Mithril-based DA solution relies on a subset of Cardano’s stake—specifically, the stake controlled by SPOs (Stake Pool Operators) running Mithril or, equivalently, the total stake actively participating in Mithril. While this still provides strong cryptographic guarantees, it does not benefit from the full economic security of Cardano’s entire stake-weighted consensus, making it theoretically more susceptible to adversarial control if a sufficiently large fraction of the Mithril-active stake colludes or becomes compromised. 

This difference in security between the Leios blob and Mithril-based data availability (DA) solutions is alleviated as more SPOs participate in Mithril. As a greater number of SPOs run Mithril nodes, a larger fraction of Cardano's total stake becomes actively involved in securing DA, thereby increasing the overall economic security of the Mithril-based approach. If Mithril were to achieve near-universal adoption among SPOs, then the total stake securing Mithril would closely approach the total active stake of Cardano, making it nearly as resilient as Cardano's full consensus mechanism. In the ideal scenario where all SPOs participate in Mithril, the Mithril DA solution would provide equivalent resilience to the Leios blob solution, as both would be backed by the same stake-weighted economic security. Therefore, the effectiveness of Mithril as a DA solution is directly proportional to its adoption rate among SPOs, and broad participation is key to minimizing any security gap between the two approaches.

\end{document}
